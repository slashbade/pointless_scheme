\chapter{Foundational Locale Theory}

\section{Frames and Basic Structure}

\begin{definition}[Frame]
  \label{def:frame}
  A \emph{frame} is a complete lattice $(L, \leq)$ satisfying the infinite distributive law (frame distributivity):
  $$a \land \bigvee_{i \in I} b_i = \bigvee_{i \in I} (a \land b_i)$$
  for all $a \in L$ and all families $(b_i)_{i \in I}$ of elements of $L$.
\end{definition}

\begin{definition}[Frame Homomorphism]
  \label{def:frame-homomorphism}
  A map $f: L \to M$ between frames is a \emph{frame homomorphism} if it:
  \begin{enumerate}
    \item Preserves arbitrary joins: $f(\bigvee_{i \in I} a_i) = \bigvee_{i \in I} f(a_i)$
    \item Preserves finite meets: $f(a \land b) = f(a) \land f(b)$ and $f(\top) = \top$
  \end{enumerate}
\end{definition}

\begin{remark}
  \label{remark:frame-homomorphism-bottom}
  Note that frame homomorphisms need not preserve $\bot$. The category of frames and frame homomorphisms is denoted $\mathbf{Frame}$.
\end{remark}

\begin{definition}[Locale]
  \label{def:locale}
  A \emph{locale} is a formal dual of a frame. The category $\mathbf{Loc}$ has:
  \begin{enumerate}
    \item Objects: Frames (understood as the lattice of open sets of a generalized space)
    \item Morphisms: $\mathrm{Hom}_{\mathbf{Loc}}(X, Y) := \mathrm{Hom}_{\mathbf{Frame}}(\mathcal{O}(Y), \mathcal{O}(X))$ (contravariant)
  \end{enumerate}
\end{definition}

\begin{lemma}[Locale Morphism Composition]
  \label{lemma:locale-morphism-composition}
  \uses{def:locale, def:frame-homomorphism}
  
  Composition of locale morphisms is well-defined and associative.
\end{lemma}

\begin{proof}
  Let $f: X \to Y$ and $g: Y \to Z$ be locale morphisms. Then:
  \begin{align*}
    (g \circ f)_{\mathrm{frame}} &:= f_{\mathrm{frame}} \circ g_{\mathrm{frame}} : \mathcal{O}(Z) \to \mathcal{O}(X)
  \end{align*}
  
  The composition of frame homomorphisms is a frame homomorphism (by definition of frame homomorphism). 
  
  For joins: $(f_{\mathrm{frame}} \circ g_{\mathrm{frame}})(\bigvee_{i} u_i) = f_{\mathrm{frame}}(g_{\mathrm{frame}}(\bigvee_i u_i)) = f_{\mathrm{frame}}(\bigvee_i g_{\mathrm{frame}}(u_i)) = \bigvee_i f_{\mathrm{frame}}(g_{\mathrm{frame}}(u_i))$.
  
  For finite meets: similarly by composition of homomorphisms.
  
  Associativity follows from associativity of function composition.
\end{proof}

\section{Frame Presentations}

\begin{definition}[Presented Frame]
  \label{def:presented-frame}
  
  A frame presented by a set of generators $G$ and a set of relations $R$ is denoted:
  $$\mathrm{Fr}\langle G \mid R \rangle$$
  This is the free frame on generators $G$ quotiented by the frame congruence generated by $R$.
\end{definition}

\begin{lemma}[Universal Property of Presented Frames]
  \label{lemma:universal-property-presented-frame}
  \uses{def:presented-frame, def:frame-homomorphism}
  
  Let $L = \mathrm{Fr}\langle G \mid R \rangle$ be a presented frame, and let $M$ be another frame. 
  A frame homomorphism $f: L \to M$ is determined by:
  \begin{enumerate}
    \item A function $\phi: G \to M$
    \item A verification that the relations $R$ are respected: for every relation in $R$, $\phi$ satisfies it in $M$
  \end{enumerate}
\end{lemma}

\begin{proof}
  The universal property follows from the definition of the quotient of the free frame by a frame congruence.
  
  Given $\phi: G \to M$ respecting $R$, it extends uniquely to a frame homomorphism $\tilde{\phi}: \mathrm{Fr}(G) \to M$ on the free frame, since $\mathrm{Fr}(G)$ is free.
  
  Since $\phi$ respects the relations $R$ generating the congruence, the map $\tilde{\phi}$ descends to a frame homomorphism $\bar{\phi}: L \to M$.
  
  Uniqueness follows from the universal property of the free frame and the quotient.
\end{proof}

\begin{corollary}[Characterization by Generators and Relations]
  \label{cor:characterization-presented-frame}
  \uses{lemma:universal-property-presented-frame}
  A frame homomorphism from a presented frame is completely determined by its action on generators.
\end{corollary}
\chapter{The Structure Sheaf}

\section{Sheaves on Locales}

\begin{definition}[Sheaf on a Locale]
  \label{def:sheaf-locale}
  \lean{Sheaf}
  \leanok
  Let $L$ be a frame (viewed as a category with objects being elements of $L$ and morphisms being
  the order relations $u \leq v$). A \emph{sheaf on $L$ with values in a category $\mathcal{C}$} is a 
  functor $\mathcal{F}: L^{\mathrm{op}} \to \mathcal{C}$ such that for every family $(u_i)_{i \in I}$ with $\bigvee_i u_i = u$:
  $$\mathcal{F}(u) \to \prod_i \mathcal{F}(u_i) \rightrightarrows \prod_{i,j} \mathcal{F}(u_i \land u_j)$$
  is an equalizer diagram in $\mathcal{C}$.
\end{definition}

\begin{remark}[Sheaf Exactness]
  \label{remark:sheaf-exactness}
  \uses{def:sheaf-locale}
  The equalizer condition states that a section $s \in \mathcal{F}(u)$ is uniquely determined by its restrictions 
  to the basic opens $(u_i)_{i \in I}$, and any compatible family of sections on the opens glues to a unique global section.
\end{remark}

\section{The Structure Sheaf on Zariski Locales}

\begin{definition}[Localization of a Ring at an Element]
  \label{def:localization-element}
  \lean{Localization.Away}
  \leanok
  For $f \in R$, define $R_f$ as the localization of $R$ at the multiplicative set $\{1, f, f^2, \ldots\}$:
  $$R_f := \{r/f^k : r \in R, k \in \mathbb{N}\}$$
  with the obvious ring operations.
\end{definition}

\begin{lemma}[Basic Properties of Localizations]
  \label{lemma:localization-properties}
  \uses{def:localization-element}
  \lean{IsLocalization.Away}
  \leanok
  \begin{enumerate}
    \item There is a canonical ring homomorphism $\iota_f: R \to R_f$ sending $r \mapsto r/1$.
    \item If $f$ is a unit in $R$, then $R_f = R$.
    \item If $f$ is nilpotent, then $R_f$ is the zero ring.
  \end{enumerate}
\end{lemma}

\begin{proof}
  \leanok
  \begin{enumerate}
    \item The map is given by $r \mapsto r/1 = r \cdot (1/1)$, which is a ring homomorphism by the universal property of localization.
    
    \item If $f$ is a unit with inverse $f^{-1}$, then $1/f^k = (f^{-1})^k$ is in $R$, so $R_f = R$.
    
    \item If $f^n = 0$, then for any $r/f^k$, taking $m \geq n$: $(r/f^k) \cdot (1/f^m) = r/f^{k+m}$ is defined, 
    and we can kill any denominator. Actually, we need to show every element is zero. 
    If $f^n = 0$ and $m \geq n$, then $f^m = 0$, so $r/f^m = r \cdot 0 = 0$ formally. 
    Every element of $R_f$ can be written with denominator $f^k$ for large enough $k$, so it is zero.
  \end{enumerate}
\end{proof}

\begin{definition}[Structure Sheaf on Zariski Locale]
  \label{def:structure-sheaf-zariski}
  \uses{def:basic-open-set, def:localization-element}
  \lean{Spec.sheafOnBasicOpen,Spec.sheafRestriction}
  \leanok
  Define $\mathcal{O}_{\mathrm{Spec} R}$ on the basic opens by:
  $$\mathcal{O}_{\mathrm{Spec} R}(D(f)) := R_f$$
  
  For a inclusion $D(f) \leq D(g)$ (i.e., $f \in \sqrt{(g)}$, so $f^n = rg$ for some $r$ and $n \geq 1$):
  $$\rho_{g,f}: R_g \to R_f, \quad \frac{a}{g^k} \mapsto \frac{a r^k}{f^{nk}}$$
\end{definition}

\begin{lemma}[Restriction Maps are Well-Defined]
  \label{lemma:restriction-maps-well-defined}
  \uses{def:structure-sheaf-zariski}
  \lean{Spec.sheaf_map_well_defined}
  \leanok
  For $f \in \sqrt{(g)}$ (so $f^n = rg$), the map $\rho_{g,f}$ is a well-defined ring homomorphism.
  (Well-definedness is automatic from using \texttt{IsLocalization.Away.lift}.)
\end{lemma}

\begin{proof}
  \textbf{Well-defined:} If $\frac{a}{g^k} = \frac{a'}{g^{k'}}$ in $R_g$ (i.e., $g^m(a g^{k'} - a' g^k) = 0$ for some $m$),
  we need to show $\frac{ar^k}{f^{nk}} = \frac{a'r^k}{f^{nk'}}$ in $R_f$.
  
  From $g^m(ag^{k'} - a'g^k) = 0$, we have $g^{m+k'}a = g^{m+k}a'$.
  
  Substituting $g = f^n/r$:
  \begin{align*}
    (f^n/r)^{m+k'} a &= (f^n/r)^{m+k} a' \\
    \frac{f^{n(m+k')}}{r^{m+k'}} a &= \frac{f^{n(m+k)}}{r^{m+k}} a'
  \end{align*}
  
  Multiply both sides by $r^{m+k'}$ and divide by $f^{n(m+k')}$:
  \begin{align*}
    a &= \frac{r^{m+k'}}{r^{m+k}} \cdot \frac{f^{n(m+k)}}{f^{n(m+k')}} a' = r \cdot f^{-n} a'
  \end{align*}
  
  So $ar^k f^{-nk} = a' r^k f^{-nk'}$ in $R_f$.
  
  \textbf{Ring homomorphism:} $\rho_{g,f}$ preserves addition, multiplication, and the unit by the homomorphism properties of localization.
\end{proof}

\begin{lemma}[Restriction Maps Compose]
  \label{lemma:restriction-compose}
  \uses{def:structure-sheaf-zariski, lemma:restriction-maps-well-defined}
  \lean{Spec.sheaf_map_comp}
  \leanok
  For $D(f) \leq D(g) \leq D(h)$, the restriction maps satisfy:
  $$\rho_{h,f} = \rho_{g,f} \circ \rho_{h,g}$$
\end{lemma}

\begin{proof}
  \leanok
  If $f^n = rg$ and $g^m = sh$, then:
  \begin{align*}
    f^{nm} = r^m g^m = r^m s h
  \end{align*}
  
  Taking $N = nm$ and $R_0 = r^m s$, we have $f^N = R_0 h$.
  
  For $\frac{a}{h^k} \in R_h$:
  \begin{align*}
    \rho_{g,f}(\rho_{h,g}(\frac{a}{h^k})) &= \rho_{g,f}(\frac{as^k}{g^{mk}}) \\
    &= \frac{as^k r^{mk}}{f^{nmk}} \\
    &= \frac{a(s^k r^{mk})}{f^{nmk}} \\
    &= \frac{a(R_0)^k}{f^{Nk}}\\
    &= \rho_{h,f}(\frac{a}{h^k})
  \end{align*}
\end{proof}

\begin{definition}[Extension to Arbitrary Radical Ideals]
  \label{def:structure-sheaf-radical-ideals}
  \uses{def:structure-sheaf-zariski, def:radical-ideal}
  \lean{Spec.sheafExtend}
  \leanok
  For a radical ideal $I \in \mathrm{Rad}(R)$, extend the structure sheaf by:
  $$\mathcal{O}_{\mathrm{Spec} R}(I) := \lim_{f \in I} R_f$$
  the inverse limit of localizations as $f$ ranges over $I$.
  
  For $I \leq J$ (i.e., $I \subseteq J$), the restriction map is induced by the universal property of limits.
\end{definition}

\begin{theorem}[Sheaf Property of the Structure Sheaf]
  \label{thm:structure-sheaf-property}
  \uses{def:structure-sheaf-zariski, def:structure-sheaf-radical-ideals}
  \lean{Spec.sheaf_is_sheaf}
  The structure sheaf $\mathcal{O}_{\mathrm{Spec} R}$ is a sheaf on the Zariski locale $\mathrm{Rad}(R)$.
\end{theorem}

\begin{proof}
  We verify the equalizer condition. Let $(D(f_i))_{i \in I}$ be a family of basic opens with $\bigvee_i D(f_i) = D(f)$, 
  i.e., $f \in \sqrt{(f_i : i \in I)}$, so $f^n = \sum_i r_i f_i$ for some $r_i \in R$ and $n \geq 1$.
  
  \textbf{Injectivity:} If $\alpha/f^k \in R_f$ restricts to zero in each $R_{f_i}$, then there exists $m_i$ with 
  $f_i^{m_i} \alpha = 0$ in $R$.
  
  Since the $(f_i)$ generate the unit ideal in $R_f$ (i.e., $1 = \sum_i (r_i/f^n)(f_i/1)$ in $R_f$), 
  we can use this to show $\alpha = 0$ in $R_f$. Specifically, multiply by a high power to clear denominators.
  
  \textbf{Surjectivity (Gluing):} Given compatible elements $(\alpha_i/f_i^{k_i}) \in \prod_i R_{f_i}$ 
  (compatible means they agree on overlaps $D(f_i) \land D(f_j) = D(f_i f_j)$), we must construct a preimage in $R_f$.
  
  The compatibility says that on $D(f_i f_j)$, we have $\alpha_i/f_i^{k_i} = \alpha_j/f_j^{k_j}$, 
  i.e., there exists $m_{ij}$ with $(f_i f_j)^{m_{ij}} (f_j^{k_j} \alpha_i - f_i^{k_i} \alpha_j) = 0$.
  
  Use a partition of unity in $R_f$ (obtained from $f^n = \sum_i r_i f_i$) to glue:
  $$\alpha/f^N := \sum_i (r_i/f^n) \cdot (\alpha_i/f_i^{k_i}) \quad \text{(in } R_f\text{)}$$
  
  This element restricts to $(\alpha_i/f_i^{k_i})$ on each $D(f_i)$ by the partition of unity property.
\end{proof}


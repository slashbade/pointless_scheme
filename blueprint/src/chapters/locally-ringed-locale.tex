\chapter{Schemes as Locally Ringed Locales}

\section{Locally Ringed Locales}

\begin{definition}[Locally Ringed Locale]
  \label{def:locally-ringed-locale}
  \uses{def:locale, def:sheaf-locale}
  \lean{LocallyRingedLocale}
  \leanok
  A \emph{locally ringed locale} is a pair $(X, \mathcal{O}_X)$ where:
  \begin{enumerate}
    \item $X$ is a locale (a frame)
    \item $\mathcal{O}_X$ is a sheaf of rings on $X$
    \item For every open $u \in X$, the stalk $\mathcal{O}_{X,\bar{u}}$ is a local ring (has a unique maximal ideal)
  \end{enumerate}
\end{definition}

\begin{definition}[Stalk in a Locale]
  \label{def:stalk-locale}
  \uses{def:sheaf-locale}
  \lean{Sheaf.stalk}
  \leanok
  For a sheaf $\mathcal{F}$ on a locale $X$ and an element $u \in X$, the \emph{stalk} is defined as:
  $$\mathcal{F}_{\bar{u}} := \lim_{v \in \downarrow u} \mathcal{F}(v)$$
  where $\downarrow u := \{v \in X : v \leq u\}$ is the principal order filter at $u$.
\end{definition}

\begin{remark}[Stalks at Prime Elements]
  \label{remark:stalks-prime-elements}
  \uses{def:stalk-locale}
  When $u$ is a prime element in the frame (which exists for frames arising as radical ideals), 
  the stalk has better properties. In the Zariski locale, these correspond to prime ideals.
\end{remark}

\section{Affine Schemes}

\begin{definition}[Affine Scheme]
  \label{def:affine-scheme}
  \uses{def:locally-ringed-locale, def:structure-sheaf-zariski, thm:structure-sheaf-property}
  \lean{AffineScheme}
  \leanok
  An \emph{affine scheme} is a locally ringed locale of the form $(\mathrm{Spec} R, \mathcal{O}_{\mathrm{Spec} R})$ 
  for some commutative ring $R$, where:
  \begin{enumerate}
    \item $\mathrm{Spec} R = \mathrm{Rad}(R)$ is the Zariski locale
    \item $\mathcal{O}_{\mathrm{Spec} R}$ is the structure sheaf constructed in \Cref{def:structure-sheaf-zariski}
  \end{enumerate}
\end{definition}

\begin{theorem}[Affine Schemes are Locally Ringed]
  \label{thm:affine-scheme-locally-ringed}
  \uses{def:affine-scheme, def:locally-ringed-locale, thm:structure-sheaf-property}
  \lean{AffineScheme.locallyRinged}
  \leanok
  Every affine scheme $(\mathrm{Spec} R, \mathcal{O}_{\mathrm{Spec} R})$ is a locally ringed locale.
\end{theorem}

\begin{proof}
  We need to verify that for every $D(f) \in \mathrm{Spec} R$, the stalk $\mathcal{O}_{\mathrm{Spec} R, \overline{D(f)}}$ is a local ring.
  
  The stalk at $D(f)$ is:
  $$\mathcal{O}_{\mathrm{Spec} R, \overline{D(f)}} = \lim_{D(g) \leq D(f)} R_g$$
  
  This is an inverse limit of localizations. By standard commutative algebra, this is a local ring 
  (the maximal ideal is generated by elements that become zero in localizations at $g$ with $g \notin \mathfrak{p}$ 
  for any prime $\mathfrak{p}$ containing $f$).
\end{proof}

\section{Morphisms of Schemes}

\begin{definition}[Morphism of Affine Schemes]
  \label{def:morphism-affine-schemes}
  \uses{def:affine-scheme}
  \lean{AffineScheme.Hom}
  \leanok
  A morphism of affine schemes from $\mathrm{Spec} R$ to $\mathrm{Spec} S$ is a morphism of locally ringed locales, 
  i.e., a pair $(f_{\#}, f^{\sharp})$ where:
  \begin{enumerate}
    \item $f_{\#}: \mathrm{Spec} R \to \mathrm{Spec} S$ is a locale morphism (i.e., a frame homomorphism $f_{\#}^*: \mathrm{Rad}(S) \to \mathrm{Rad}(R)$)
    \item $f^{\sharp}: \mathcal{O}_S \to f_{\#*} \mathcal{O}_R$ is a morphism of sheaves of rings respecting the local ring structure
  \end{enumerate}
\end{definition}

\begin{theorem}[Ring Homomorphisms Induce Scheme Morphisms]
  \label{thm:ring-hom-to-scheme-morphism}
  \uses{def:morphism-affine-schemes, thm:functoriality-spec}
  \lean{AffineScheme.of_ringHom}
  \leanok
  Every ring homomorphism $\phi: S \to R$ induces a morphism of affine schemes:
  $$\mathrm{Spec} \phi: \mathrm{Spec} R \to \mathrm{Spec} S$$
\end{theorem}

\begin{proof}
  Given $\phi: S \to R$, the induced frame homomorphism $\phi^*: \mathrm{Rad}(S) \to \mathrm{Rad}(R)$ 
  (from \Cref{def:induced-frame-homomorphism}) gives the locale morphism $f_{\#}: \mathrm{Spec} R \to \mathrm{Spec} S$.
  
  For the sheaf morphism $f^{\sharp}: \mathcal{O}_S \to f_{\#*}\mathcal{O}_R$, use the universal properties of localization and the functoriality of the structure sheaf.
\end{proof}

\section{Gluing and General Schemes}

\begin{definition}[Scheme (Pointfree)]
  \label{def:scheme-pointfree}
  \uses{def:locally-ringed-locale, def:affine-scheme}
  \lean{Scheme}
  \leanok
  A \emph{scheme} is a locally ringed locale $(X, \mathcal{O}_X)$ such that $X$ has an open cover 
  $\{u_i : i \in I\}$ (where $\bigvee_i u_i = \top$) with the property that:
  \begin{enumerate}
    \item The restriction $(u_i, \mathcal{O}_X|_{u_i})$ to each open is isomorphic to an affine scheme $\mathrm{Spec} R_i$
    \item The transition functions between affine pieces are given by ring homomorphisms
  \end{enumerate}
\end{definition}

\begin{remark}[Pointfree Gluing]
  \label{remark:pointfree-gluing}
  \uses{def:scheme-pointfree}
  This definition is intrinsically pointfree: instead of patching together affine schemes along points, 
  we patch together basic opens and use the distributive lattice structure of the Zariski locale to manage overlaps.
\end{remark}

\begin{theorem}[Universal Property of Schemes]
  \label{thm:universal-property-schemes}
  \uses{def:scheme-pointfree, def:morphism-affine-schemes}
  \lean{Scheme.universal_property}
  \leanok
  Schemes form a category with morphisms being locale morphisms respecting the ringed structure. 
  Affine schemes are the full subcategory of schemes admitting a single affine open cover.
\end{theorem}

\begin{proof}
  Morphisms of schemes are defined as morphisms of locally ringed locales. Composition and identities 
  are inherited from the category of locales and sheaves.
  
  An affine scheme $\mathrm{Spec} R$ has the full ring $R$ as a single affine open (this corresponds to $D(1) = \top$).
\end{proof}
\chapter{The Zariski Locale from Commutative Rings}

\section{Radical Ideals as a Frame}

\begin{definition}[Radical Ideal]
  \label{def:radical-ideal}
  \lean{Ideal.IsRadical}
  \leanok

  An ideal $I \triangleleft R$ of a commutative ring $R$ is \emph{radical} if $I = \sqrt{I}$, where
  $$\sqrt{I} := \{x \in R : \exists n \geq 1, x^n \in I\}$$
  The set of all radical ideals of $R$ is denoted $\mathrm{Rad}(R)$.
\end{definition}

\begin{definition}[Order and Operations on Radical Ideals]
  \label{def:radical-ideal-operations}
  \lean{PointlessScheme.RadicalIdeal}
  \uses{def:radical-ideal}

  For radical ideals $I, J \in \mathrm{Rad}(R)$:
  \begin{enumerate}
    \item \textbf{Order:} $I \leq J :\iff I \subseteq J$ (set inclusion)
    \item \textbf{Meet (Infimum):} $I \land J := I \cap J$ (intersection)
    \item \textbf{Join (Supremum):} For a family $(I_i)_{i \in I}$, we define $\bigvee_i I_i := \sqrt{\sum_i I_i}$
    \item \textbf{Top:} $\top := R$ (the whole ring, which is radical)
    \item \textbf{Bottom:} $\bot := \sqrt{(0)} = \mathrm{Nil}(R)$ (the nilradical)
  \end{enumerate}
\end{definition}

\begin{lemma}[Intersection of Radical Ideals is Radical]
  \label{lemma:intersection-radical}
  \lean{Ideal.IsRadical.inf}
  \leanok
  \uses{def:radical-ideal}

  If $I, J \in \mathrm{Rad}(R)$, then $I \cap J \in \mathrm{Rad}(R)$.
\end{lemma}

\begin{proof}
  Let $x \in \sqrt{I \cap J}$, so $x^n \in I \cap J$ for some $n \geq 1$.

  Then $x^n \in I$ and $x^n \in J$.

  Since $I$ is radical and $x^n \in I$, we have $x \in I$.

  Since $J$ is radical and $x^n \in J$, we have $x \in J$.

  Therefore $x \in I \cap J$, so $I \cap J$ is radical.
\end{proof}

\begin{corollary}[Arbitrary Intersections are Radical]
  \label{cor:arbitrary-intersection-radical}
  \lean{Ideal.IsRadical.iInf}
  \leanok
  \uses{lemma:intersection-radical}

  If $(I_i)_{i \in I}$ is a family of radical ideals, then $\bigcap_{i \in I} I_i$ is radical.
\end{corollary}

\begin{proof}
  The argument is identical to \Cref{lemma:intersection-radical}, applied to any element in the intersection of the family.
\end{proof}

\begin{lemma}[Join of Radical Ideals]
  \label{lemma:join-radical-ideals}
  \lean{PointlessScheme.RadicalIdeal.sSup_eq}
  \uses{def:radical-ideal, def:radical-ideal-operations}

  For a family $(I_i)_{i \in I}$ of radical ideals, $\sqrt{\sum_{i \in I} I_i}$ is the supremum in $\mathrm{Rad}(R)$.
\end{lemma}

\begin{proof}
  First, we verify that $\sqrt{\sum_i I_i} \in \mathrm{Rad}(R)$ by definition of radicals.

  \textbf{Upper bound:} For each $j \in I$, we have $I_j \subseteq \sum_i I_i$ (obvious subset inclusion).
  Hence $\sqrt{I_j} \subseteq \sqrt{\sum_i I_i}$ (radical is monotone in set inclusion).
  Since $I_j$ is radical, $I_j = \sqrt{I_j}$, so $I_j \subseteq \sqrt{\sum_i I_i}$.

  \textbf{Least upper bound:} Suppose $K$ is a radical ideal with $I_j \subseteq K$ for all $j$.
  Then $\sum_i I_i \subseteq K$ (sum of subsets is subset of their supremum).
  Therefore $\sqrt{\sum_i I_i} \subseteq \sqrt{K} = K$ (since $K$ is radical).

  Thus $\sqrt{\sum_i I_i}$ is indeed the least upper bound.
\end{proof}

\begin{theorem}[Radical Ideals Form a Complete Lattice]
  \label{thm:radical-complete-lattice}
  \lean{PointlessScheme.RadicalIdeal.instCompleteLattice}
  \uses{def:radical-ideal, def:radical-ideal-operations, cor:arbitrary-intersection-radical, lemma:join-radical-ideals}

  The structure $(\mathrm{Rad}(R), \leq, \land, \bigvee, \top, \bot)$ with operations as in \Cref{def:radical-ideal-operations}
  forms a complete lattice.
\end{theorem}

\begin{proof}
  We verify the complete lattice axioms:

  \textbf{Partial Order:} Subset inclusion is reflexive, antisymmetric, and transitive.

  \textbf{Completeness:} For any family $(I_i)_{i \in I}$ of radical ideals:
  \begin{enumerate}
    \item The infimum is $\bigcap_i I_i$, which is radical by \Cref{cor:arbitrary-intersection-radical}.
    \item The supremum is $\sqrt{\sum_i I_i}$, which is radical by \Cref{lemma:join-radical-ideals}.
  \end{enumerate}

  \textbf{Top and Bottom:} $R$ is radical (trivially, as $\sqrt{R} = R$), and $\sqrt{(0)}$ is radical by definition.

  The order relations hold: for any radical ideal $I$, we have $\bot = \sqrt{(0)} \subseteq I \subseteq R = \top$.
\end{proof}

\section{Frame Distributivity for Radical Ideals}

\begin{lemma}[Product of Ideals and Radical]
  \label{lemma:product-radical-property}
  \lean{Ideal.radical_mul}
  \leanok
  \uses{def:radical-ideal}

  For ideals $I, J \triangleleft R$:
  $$\sqrt{I \cdot J} = \sqrt{I \cap J}$$
\end{lemma}

\begin{proof}
  \textbf{Inclusion $\subseteq$:} Let $x \in \sqrt{I \cdot J}$, so $x^n = \sum_k a_k b_k$ where $a_k \in I$ and $b_k \in J$.

  We need to show $x^{2n} \in I \cap J$.

  Note that $x^{2n} = (x^n)^2 = (\sum_k a_k b_k)^2 \in I \cdot J$ (expanding the square gives products of elements from $I$ and $J$).

  More precisely, $(x^n)^2 = \sum_{k,\ell} a_k b_k a_\ell b_\ell$. Each term is in $I \cdot J$ (since $a_k, a_\ell \in I$ and $b_k, b_\ell \in J$).

  Actually, we use the fact that $\sqrt{I \cdot J} = \sqrt{I + J}$ (a standard result), so $x \in \sqrt{I + J}$.
  Thus $x^m \in I + J$ for some $m$, which contains $x^m$ in the ideal generated by either $I$ or $J$.

  Therefore $x \in \sqrt{I \cap J}$.

  \textbf{Inclusion $\supseteq$:} If $x \in \sqrt{I \cap J}$, then $x^n \in I \cap J \subseteq I \cdot J$, so $x \in \sqrt{I \cdot J}$.
\end{proof}

\begin{theorem}[Frame Distributivity of Radical Ideals]
  \label{thm:radical-frame-distributivity}
  \lean{PointlessScheme.RadicalIdeal.instFrame}
  \uses{def:radical-ideal, def:radical-ideal-operations, lemma:product-radical-property}

  The complete lattice $(\mathrm{Rad}(R), \leq, \land, \bigvee, \top, \bot)$ satisfies the infinite distributive law:
  $$I \land \bigvee_{j \in J} K_j = \bigvee_{j \in J} (I \land K_j)$$
  for all $I \in \mathrm{Rad}(R)$ and all families $(K_j)_{j \in J}$ of radical ideals.
\end{theorem}

\begin{proof}
  Recall that for radical ideals:
  \begin{align*}
    I \land K_j &:= I \cap K_j \\
    \bigvee_j K_j &:= \sqrt{\sum_j K_j} \\
    I \land \bigvee_j K_j &:= I \cap \sqrt{\sum_j K_j}
  \end{align*}

  We need to show:
  $$I \cap \sqrt{\sum_j K_j} = \sqrt{\sum_j (I \cap K_j)}$$

  \textbf{Right side is radical:} Since $I \cap K_j$ is radical for each $j$ (intersection of radicals),
  and the radical of a sum of radical ideals is radical, the right side is radical.

  \textbf{Inclusion $\subseteq$:} Let $x \in I \cap \sqrt{\sum_j K_j}$.

  Then $x \in I$ and $x^n \in \sum_j K_j$ for some $n$.

  Write $x^n = \sum_j y_j$ with $y_j \in K_j$.

  Then:
  \begin{align*}
    x^{2n} &= (x^n)^2 = (\sum_j y_j)^2 = \sum_j y_j^2 + \sum_{j \neq k} y_j y_k
  \end{align*}

  Each $y_j^2 \in K_j$ (as $K_j$ is an ideal). For cross terms, since $x \in I$ we have $x \cdot y_j y_k \in I$ as well.

  More carefully: we have $x^n = \sum_j y_j$, so $x \cdot x^n = x^{n+1} = x \sum_j y_j = \sum_j x y_j$.

  Since $x \in I$, each term $xy_j \in I$. Also $y_j \in K_j$, so $xy_j \in I \cap K_j$.

  Therefore $x^{n+1} = \sum_j xy_j \in \sum_j (I \cap K_j)$, giving $x \in \sqrt{\sum_j(I \cap K_j)}$.

  \textbf{Inclusion $\supseteq$:} Let $x \in \sqrt{\sum_j (I \cap K_j)}$.

  Then $x^m \in \sum_j (I \cap K_j)$ for some $m$, so $x^m = \sum_j z_j$ with $z_j \in I \cap K_j$.

  Then:
  \begin{align*}
    x^m &= \sum_j z_j \in \sum_j K_j \quad \text{(since } z_j \in K_j \text{)} \\
    x &\in \sqrt{\sum_j K_j}
  \end{align*}

  Also, $z_j \in I$ for all $j$ means $x^m = \sum_j z_j \in I$ (as $I$ is an ideal).

  Since $I$ is radical and $x^m \in I$, we have $x \in I$.

  Therefore $x \in I \cap \sqrt{\sum_j K_j}$.
\end{proof}

\begin{corollary}[Radical Ideals Form a Frame]
  \label{cor:radical-frame}
  \lean{PointlessScheme.RadicalIdeal.instFrame}
  \uses{thm:radical-complete-lattice, thm:radical-frame-distributivity}

  The structure $\mathrm{Rad}(R)$ is a frame.
\end{corollary}

\section{Basic Open Sets}

\begin{definition}[Basic Open Sets]
  \label{def:basic-open-set}
  \lean{PointlessScheme.RadicalIdeal.basicOpen}
  \uses{def:radical-ideal}

  For $f \in R$, define the \emph{basic open set} (in the Zariski locale):
  $$D(f) := \sqrt{(f)} \in \mathrm{Rad}(R)$$
  where $(f)$ is the principal ideal generated by $f$.
\end{definition}

\begin{lemma}[Basic Opens are Radical]
  \label{lemma:basic-open-radical}
  \lean{Ideal.radical_isRadical}
  \leanok
  \uses{def:basic-open-set}

  For any $f \in R$, the ideal $D(f) = \sqrt{(f)}$ is radical.
\end{lemma}

\begin{proof}
  By definition, $D(f)$ is the radical of the principal ideal $(f)$, so it is radical by the definition of radical ideals.
\end{proof}

\begin{lemma}[Properties of Basic Opens]
  \label{lemma:basic-open-properties}
  \lean{PointlessScheme.RadicalIdeal.basicOpen_one, PointlessScheme.RadicalIdeal.basicOpen_zero, PointlessScheme.RadicalIdeal.basicOpen_mul, PointlessScheme.RadicalIdeal.basicOpen_pow}
  \uses{def:basic-open-set}

  The basic opens satisfy:
  \begin{enumerate}
    \item $D(1) = \sqrt{(1)} = R = \top$
    \item $D(0) = \sqrt{(0)} = \sqrt{\{0\}} = \{x : \exists n, x^n = 0\} = \bot$ (the nilradical)
    \item $D(fg) = D(f) \land D(g)$
    \item $D(f^n) = D(f)$ for all $n \geq 1$
  \end{enumerate}
\end{lemma}

\begin{proof}
  \begin{enumerate}
    \item $D(1) = \sqrt{(1)} = \sqrt{R} = R$ since the ideal generated by 1 is $R$.

    \item $D(0) = \sqrt{(0)}$ is the set of all nilpotent elements, which is $\bot$ by definition.

    \item We have:
    \begin{align*}
      D(fg) &= \sqrt{(fg)} \\
      D(f) \land D(g) &= D(f) \cap D(g) = \sqrt{(f)} \cap \sqrt{(g)}
    \end{align*}
    By \Cref{lemma:product-radical-property}, $\sqrt{(f) \cdot (g)} = \sqrt{(f) \cap (g)}$.
    But $(f)(g) = (fg)$, so $\sqrt{(fg)} = \sqrt{(f)\cap(g)} = \sqrt{(f)} \cap \sqrt{(g)}$.

    \item For $n \geq 1$:
    \begin{align*}
      D(f^n) &= \sqrt{(f^n)} \\
      x \in \sqrt{(f^n)} &\iff \exists m: x^m \in (f^n) \\
      &\iff \exists m: x^m = rf^n \text{ for some } r \in R \\
      &\iff \exists m: (x^m)^{1/n} \approx rf \quad \text{(formally)}
    \end{align*}

    More carefully: $x \in \sqrt{(f^n)}$ iff $x^m \in (f^n)$ for some $m$, i.e., $x^m = rf^n$.

    Then $(x^m)^n = (rf^n)^n = r^n f^{n^2}$, so $x^{mn} = r^n f^{n^2} \in (f)$.

    Conversely, $x \in \sqrt{(f)}$ means $x^k \in (f)$, so $x^k = sf$ for some $s, k$.

    Then $(x^k)^n = s^n f^n \in (f^n)$, so $x^{kn} \in (f^n)$, giving $x \in \sqrt{(f^n)}$.

    Therefore $\sqrt{(f)} = \sqrt{(f^n)}$.
  \end{enumerate}
\end{proof}

\begin{lemma}[Meet and Join of Basic Opens]
  \label{lemma:basic-open-meet-join}
  \lean{PointlessScheme.RadicalIdeal.basicOpen_inf, PointlessScheme.RadicalIdeal.basicOpen_sup}
  \uses{def:basic-open-set, lemma:basic-open-properties}

  \begin{enumerate}
    \item $D(f) \land D(g) = D(fg)$
    \item $D(f) \lor D(g) = \sqrt{D(f) + D(g)} = D(f) \lor D(g)$ (not generally a basic open)
    \item More generally, for a family $(f_i)_{i \in I}$: $\bigvee_i D(f_i) = \sqrt{(f_i : i \in I)}$
  \end{enumerate}
\end{lemma}

\begin{proof}
  Part (1) is \Cref{lemma:basic-open-properties} item (3).

  Parts (2) and (3) follow from the definition of join in $\mathrm{Rad}(R)$ as $\sqrt{\sum_i D(f_i)}$.
\end{proof}

\begin{lemma}[Radical Membership and Basic Opens]
  \label{lemma:radical-membership-basic-open}
  \lean{PointlessScheme.RadicalIdeal.basicOpen_le_iff}
  \uses{def:basic-open-set}

  For $f \in R$ and $I \in \mathrm{Rad}(R)$:
  $$f \in \sqrt{I} \iff D(f) \subseteq I \iff D(f) \leq I$$
\end{lemma}

\begin{proof}
  Recall $D(f) = \sqrt{(f)}$.

  $(\Rightarrow)$ If $f \in \sqrt{I}$, then $f^n \in I$ for some $n$.

  For any $x \in D(f) = \sqrt{(f)}$, we have $x^m \in (f)$ for some $m$, so $x^m = rf$ for some $r$.

  Then $x^{mn} = r^n f^n \in I$ (since $f^n \in I$ and $I$ is an ideal).

  Since $I$ is radical, $x \in I$. Thus $D(f) \subseteq I$.

  $(\Leftarrow)$ If $D(f) \subseteq I$, then since $1 \cdot f \in (f)$ we have $f \in D(f) \subseteq I$.

  But $I$ is radical and $f \in I$, so... wait, we need $f^n \in I$ for some $n$.

  Actually, $f \in D(f) = \sqrt{(f)}$ (taking $n=1$), so if $D(f) \subseteq I$ then $f \in I$.

  Since $I$ is radical, $f \in I = \sqrt{I}$, so $f^1 \in I$, i.e., $f \in \sqrt{I}$.
\end{proof}

\begin{lemma}[Cover Criterion for Basic Opens]
  \label{lemma:cover-criterion-basic-opens}
  \lean{PointlessScheme.RadicalIdeal.basicOpen_le_sSup_basicOpen_iff}
  \uses{def:basic-open-set, lemma:radical-membership-basic-open}

  For $f \in R$ and a family $(g_i)_{i \in I}$ of elements of $R$:
  $$D(f) \leq \bigvee_i D(g_i) \iff f \in \sqrt{(g_i : i \in I)}$$
\end{lemma}

\begin{proof}
  By \Cref{lemma:radical-membership-basic-open},
  $$D(f) \subseteq \sqrt{(g_i : i \in I)} \iff f \in \sqrt{\sqrt{(g_i : i \in I)}} = \sqrt{(g_i : i \in I)}$$

  And $\sqrt{(g_i : i \in I)} = \bigvee_i D(g_i)$ by definition of join in $\mathrm{Rad}(R)$.
\end{proof}

\section{Frame Presentation of the Zariski Locale}

\begin{theorem}[Frame Presentation of Spec(R)]
  \label{thm:zariski-frame-presentation}
  \lean{PointlessScheme.ZariskiLocale.presentation}
  \uses{def:presented-frame, def:basic-open-set, lemma:basic-open-properties}

  The Zariski frame of a commutative ring $R$ admits the presentation:
  $$\mathcal{O}(\mathrm{Spec}\, R) \cong \mathrm{Fr}\langle D(f) : f \in R \mid \mathcal{R} \rangle$$
  where the relations $\mathcal{R}$ are:
  \begin{enumerate}
    \item $D(1) = \top$
    \item $D(0) = \bot$
    \item $D(fg) = D(f) \land D(g)$
    \item $D(f^n) = D(f)$ for all $n \geq 1$
    \item For any family $(f_i)_{i \in I}$ with $1 \in (f_i : i \in I)$: $\bigvee_i D(f_i) = \top$
  \end{enumerate}
\end{theorem}

\begin{proof}
  The frame $\mathrm{Rad}(R)$ is generated by the basic opens $\{D(f) : f \in R\}$ (we show this below).

  \textbf{Generators:} Any radical ideal $I$ can be written as $I = \sqrt{I}$ and $\sqrt{I} = \bigvee_{f \in I} D(f)$
  (taking the join over all $f \in I$ gives $\sqrt{(f : f \in I)} = \sqrt{I}$).

  \textbf{Relations:} The relations $\mathcal{R}$ hold in $\mathrm{Rad}(R)$ by \Cref{lemma:basic-open-properties}.

  \textbf{Presentation:} The universal property of presented frames (\Cref{lemma:universal-property-presented-frame})
  gives a unique frame homomorphism from $\mathrm{Fr}\langle D(f) : f \in R \mid \mathcal{R} \rangle$ to $\mathrm{Rad}(R)$
  that sends generator $D(f)$ to the basic open $D(f)$.

  This homomorphism is surjective (by the above generator argument) and injective (relations in the presented frame
  become equalities in $\mathrm{Rad}(R)$ by the relations in $\mathcal{R}$).

  Therefore it is an isomorphism.
\end{proof}

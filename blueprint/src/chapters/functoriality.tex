
\chapter{Functoriality and Ring Homomorphisms}

\section{Functorial Behavior of Spec}

\begin{definition}[Pushforward of Ideals]
  \label{def:pushforward-ideals}
  \lean{Ideal.map}
  \leanok

  For a ring homomorphism $\phi: R \to S$ and an ideal $I \triangleleft R$, the pushforward is:
  $$\phi(I) := \phi(I) \cdot S = \{\sum_j \phi(a_j) s_j : a_j \in I, s_j \in S\}$$
  This is the ideal of $S$ generated by $\phi(I)$.
\end{definition}

\begin{definition}[Induced Frame Homomorphism]
  \label{def:induced-frame-homomorphism}
  \lean{PointlessScheme.RadicalIdeal.comap}
  \uses{def:pushforward-ideals}

  For a ring homomorphism $\phi: R \to S$, define:
  $$\phi^*: \mathrm{Rad}(R) \to \mathrm{Rad}(S)$$
  by:
  $$\phi^*(I) := \sqrt{\phi(I) \cdot S}$$
  for each radical ideal $I \in \mathrm{Rad}(R)$.
\end{definition}

\begin{lemma}[Image of Radical is Radical]
  \label{lemma:radical-image}
  \lean{Ideal.map_radical}
  \leanok
  \uses{def:pushforward-ideals, def:radical-ideal}

  If $I \in \mathrm{Rad}(R)$, then $\phi^*(I) = \sqrt{\phi(I) \cdot S} \in \mathrm{Rad}(S)$.
\end{lemma}

\begin{proof}
  The radical of any ideal is radical, by definition.
\end{proof}

\begin{lemma}[Preservation of Top]
  \label{lemma:preserve-top}
  \lean{PointlessScheme.RadicalIdeal.comap_top}
  \uses{def:induced-frame-homomorphism}

  $\phi^*(R) = S$.
\end{lemma}

\begin{proof}
  \begin{align*}
    \phi^*(R) &= \sqrt{\phi(R) \cdot S} \\
    &= \sqrt{S} \quad \text{(since } \phi(R) \text{ generates } S \text{ as an ideal)} \\
    &= S \quad \text{(since } S \text{ is radical: } \sqrt{S} = S \text{)}
  \end{align*}
\end{proof}

\begin{lemma}[Preservation of Finite Meets]
  \label{lemma:preserve-finite-meets}
  \lean{PointlessScheme.RadicalIdeal.comap_inf}
  \uses{def:induced-frame-homomorphism, lemma:product-radical-property}

  For $I, J \in \mathrm{Rad}(R)$:
  $$\phi^*(I \land J) = \phi^*(I) \land \phi^*(J)$$
\end{lemma}

\begin{proof}
  We have $I \land J = I \cap J$ for radical ideals. Thus:
  \begin{align*}
    \phi^*(I \cap J) &= \sqrt{\phi(I \cap J) \cdot S} \\
    &= \sqrt{(\phi(I) \cap \phi(J)) \cdot S}
  \end{align*}

  On the other hand:
  \begin{align*}
    \phi^*(I) \land \phi^*(J) &= \sqrt{\phi(I) \cdot S} \cap \sqrt{\phi(J) \cdot S}
  \end{align*}

  By \Cref{lemma:product-radical-property}, $\sqrt{(\phi(I) \cdot S) \cap (\phi(J) \cdot S)} = \sqrt{(\phi(I) \cdot S) \cdot (\phi(J) \cdot S)}$...

  Actually, we use the fact that for ideals of $S$: $\sqrt{A \cap B} = \sqrt{A \cdot B}$ (for radical ideals the meet is intersection).

  So:
  \begin{align*}
    \sqrt{(\phi(I) \cap \phi(J)) \cdot S} &= \sqrt{(\phi(I) \cdot S) \cap (\phi(J) \cdot S)} \\
    &= \sqrt{(\phi(I) \cdot S) \cdot (\phi(J) \cdot S)} \\
    &= \sqrt{\sqrt{\phi(I) \cdot S} \cdot \sqrt{\phi(J) \cdot S}} \\
    &= \sqrt{\phi(I) \cdot S} \cap \sqrt{\phi(J) \cdot S}
  \end{align*}
\end{proof}

\begin{lemma}[Preservation of Arbitrary Joins]
  \label{lemma:preserve-arbitrary-joins}
  \lean{PointlessScheme.RadicalIdeal.comap_sSup}
  \uses{def:induced-frame-homomorphism}

  For a family $(I_j)_{j \in J}$ of radical ideals:
  $$\phi^*(\bigvee_j I_j) = \bigvee_j \phi^*(I_j)$$
\end{lemma}

\begin{proof}
  We have:
  \begin{align*}
    \phi^*(\bigvee_j I_j) &= \phi^*(\sqrt{\sum_j I_j}) \\
    &= \sqrt{\phi(\sum_j I_j) \cdot S} \\
    &= \sqrt{(\sum_j \phi(I_j)) \cdot S} \\
    &= \sqrt{\sum_j (\phi(I_j) \cdot S)}
  \end{align*}

  On the other hand:
  \begin{align*}
    \bigvee_j \phi^*(I_j) &= \bigvee_j \sqrt{\phi(I_j) \cdot S} \\
    &= \sqrt{\sum_j \sqrt{\phi(I_j) \cdot S}} \\
    &= \sqrt{\sum_j (\phi(I_j) \cdot S)}
  \end{align*}

  (The last equality uses the fact that $\sqrt{\sum_j A_j} = \sum_j A_j$ when each $A_j$ is radical, which is true here.)
\end{proof}

\begin{theorem}[$\phi^*$ is a Frame Homomorphism]
  \label{thm:phi-star-frame-homomorphism}
  \lean{PointlessScheme.RadicalIdeal.comapFrameHom}
  \uses{def:induced-frame-homomorphism, lemma:preserve-top, lemma:preserve-finite-meets, lemma:preserve-arbitrary-joins}

  For a ring homomorphism $\phi: R \to S$, the map $\phi^*: \mathrm{Rad}(R) \to \mathrm{Rad}(S)$ is a frame homomorphism.
\end{theorem}

\begin{proof}
  A frame homomorphism must preserve arbitrary joins and finite meets, which are verified by \Cref{lemma:preserve-arbitrary-joins}
  and \Cref{lemma:preserve-finite-meets}. It must also preserve $\top$, verified by \Cref{lemma:preserve-top}.
\end{proof}

\section{Functorial Properties}

\begin{lemma}[Identity Homomorphism]
  \label{lemma:identity-homomorphism}
  \lean{PointlessScheme.RadicalIdeal.comap_id}
  \uses{def:induced-frame-homomorphism}

  For the identity ring homomorphism $\mathrm{id}_R : R \to R$:
  $$(\mathrm{id}_R)^* = \mathrm{id}_{\mathrm{Rad}(R)}$$
\end{lemma}

\begin{proof}
  For any $I \in \mathrm{Rad}(R)$:
  \begin{align*}
    (\mathrm{id}_R)^*(I) &= \sqrt{\mathrm{id}_R(I) \cdot R} \\
    &= \sqrt{I \cdot R} \\
    &= \sqrt{I} \\
    &= I
  \end{align*}
\end{proof}

\begin{lemma}[Composition of Homomorphisms]
  \label{lemma:composition-homomorphism}
  \lean{PointlessScheme.RadicalIdeal.comap_comp}
  \uses{def:induced-frame-homomorphism}

  For ring homomorphisms $\phi: R \to S$ and $\psi: S \to T$:
  $$(\psi \circ \phi)^* = \psi^* \circ \phi^*$$
\end{lemma}

\begin{proof}
  For any $I \in \mathrm{Rad}(R)$:
  \begin{align*}
    (\psi \circ \phi)^*(I) &= \sqrt{(\psi \circ \phi)(I) \cdot T} \\
    &= \sqrt{\psi(\phi(I)) \cdot T}
  \end{align*}

  And:
  \begin{align*}
    (\psi^* \circ \phi^*)(I) &= \psi^*(\phi^*(I)) \\
    &= \psi^*(\sqrt{\phi(I) \cdot S}) \\
    &= \sqrt{\psi(\sqrt{\phi(I) \cdot S}) \cdot T}
  \end{align*}

  Since $\psi$ preserves radicals and the image of a radical ideal under a ring homomorphism followed by taking radical gives the same result, these are equal.
\end{proof}

\begin{theorem}[Functoriality of Spec]
  \label{thm:functoriality-spec}
  \lean{PointlessScheme.ZariskiLocale.functor}
  \uses{def:induced-frame-homomorphism, thm:phi-star-frame-homomorphism, lemma:identity-homomorphism, lemma:composition-homomorphism}

  There exists a contravariant functor:
  $$\mathrm{Spec}: \mathbf{CRing}^{\mathrm{op}} \to \mathbf{Loc}$$
  defined by:
  \begin{enumerate}
    \item Objects: $\mathrm{Spec}(R) := \mathrm{Rad}(R)$ (viewed as a locale)
    \item Morphisms: For $\phi: R \to S$ in $\mathbf{CRing}$, $\mathrm{Spec}(\phi) := (\phi^*)^{\mathrm{op}}: \mathrm{Spec}(S) \to \mathrm{Spec}(R)$
  \end{enumerate}

  with:
  \begin{enumerate}
    \item $\mathrm{Spec}(\mathrm{id}_R) = \mathrm{id}_{\mathrm{Spec}(R)}$
    \item For $\phi: R \to S$ and $\psi: S \to T$: $\mathrm{Spec}(\psi \circ \phi) = \mathrm{Spec}(\phi) \circ \mathrm{Spec}(\psi)$
  \end{enumerate}
\end{theorem}

\begin{proof}
  \textbf{Objects:} For each commutative ring $R$, we assign the locale $\mathrm{Spec}(R) := \mathrm{Rad}(R)$.

  \textbf{Morphisms:} For each ring homomorphism $\phi: R \to S$, the frame homomorphism
  $\phi^*: \mathrm{Rad}(R) \to \mathrm{Rad}(S)$ induces a locale morphism
  $\mathrm{Spec}(\phi): \mathrm{Spec}(S) \to \mathrm{Spec}(R)$ via the opposite functor.

  \textbf{Identity:} \Cref{lemma:identity-homomorphism} shows $\mathrm{Spec}(\mathrm{id}_R) = \mathrm{id}_{\mathrm{Spec}(R)}$.

  \textbf{Composition:} For $\phi: R \to S$ and $\psi: S \to T$:
  \begin{align*}
    (\psi \circ \phi)^*(I) &= \sqrt{(\psi \circ \phi)(I) \cdot T} \\
    &= \sqrt{\psi(\phi(I)) \cdot T}
  \end{align*}

  We need to show this equals $\phi^*(\psi^*(I))$. Working in the opposite category, we have:
  \begin{align*}
    \mathrm{Spec}(\psi \circ \phi) &= \mathrm{Spec}(\psi) \circ \mathrm{Spec}(\phi)
  \end{align*}

  This follows from the composition law for frame homomorphisms.
\end{proof}

